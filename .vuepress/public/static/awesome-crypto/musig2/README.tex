%! TEX program=xelatex
\documentclass[a4paper,10pt]{article}

\usepackage[ruled,linesnumbered,vlined]{algorithm2e}
\usepackage{amssymb}
\usepackage{amsmath}
\usepackage{amsthm}
\usepackage{cite}
\usepackage{ctex}
\usepackage{enumitem}
\usepackage[top=1cm,bottom=2cm]{geometry}
\usepackage[colorlinks,
  linkcolor=red,
  anchorcolor=blue,
  citecolor=blue
]{hyperref}
\usepackage{listings}
\usepackage{url}

\lstset{
  basicstyle=\small,
  keywordstyle=\ttfamily,
  identifierstyle=\ttfamily,
  stringstyle=\ttfamily,
  showstringspaces=false
}

\newcommand{\bG}{\mathbb{G}}
\newcommand{\bZ}{\mathbb{Z}}

\newtheorem{definition}{\bf 定义}[section]

\title{MuSig2:简洁的 2 轮 Schnorr 多签名}
\author{sammyne}
\date{\today}

\begin{document}
\maketitle

\section{背景}

\subsection{对旧版 MuSig 的攻击}
2019 年 Drijvers 等人~\cite{2019On} 发现了一种对旧版 MuSig ~\cite{cryptoeprint:2018:068:20180118:124757} 等两轮模式多签名的攻击方法。攻击的底层原理是解决泛化生日问题的 Wagner 算法~\cite{wagner2002generalized}。

\begin{definition}[泛化生日问题]
  给定常量 \(t\in \bZ_p\),整数 \(k_m\) 以及一个随机预言机 \(H: \bZ_p\rightarrow \{0,1\}^n\),通过 \(k_m\) 次查询找出满足 \(\sum_{k=1}^{k_m}H(q_k)=t\) 的集合 \(\{q_1,\dots,q_{k_m}\}\)。问题的解决复杂度,\(k_m=1\)时等价于找出哈希原像,\(k_m=2\)时等价于找出哈希碰撞,\(k_m\) 增大时,问题难度会神奇地变得更低。Wagner 等人~\cite{wagner2002generalized} 给出一个在不限定 \(k_m\) 条件下的次指数级别算法。
\end{definition}

Wagner 算法的具体攻击流程如下:敌手同时开启 \(k_m\) 个签名会话,会话过程敌手扮演公钥为 \(X_2=g^{x_2}\) 的签名方,从公钥为 \(X_1=g^{x_1}\) 的诚实签名方获得共 \(k_m\) 个 nonces \(R_1^1,\dots,R_1^{k_m}\)。令 \(\tilde{X}=X_1^{a_1}X_2^{a_2}\)(其中 \(a_i=H(\langle X_1,X_2\rangle,X_i)\)~\cite{cryptoeprint:2018:068:20180118:124757})表示聚合所得公钥。给定伪造的消息 \(m^*\),敌手计算 \(R^*=\prod_{k=1}^{k_m} R_1^k\),然后借助 Wagner 算法找出满足以下条件的 \(R_2^k\)

\begin{equation}\label{fake-R2}
  \sum_{k=1}^{k_m} \underbrace{H_{sig}(\tilde{X},R_1^k R_2^k,m^k)}_{c^k} = \underbrace{H_{sig}(\tilde{X},R^*,m^*)}_{c^*}
\end{equation}

诚实签名方收到 \(R_2^k\) 后会反馈 \(s_1^k=r_1^k+c^k\cdot a_1x_1\)。令 \(r^*=\sum_{k=1}^{k_m}r_1^k=DL(R^*)\),敌手借此可得
\[
  s_1^* = \sum_{k=1}^{k_m} s_1^k = \sum_{k=1}^{k_m} r_1^k + \left(\sum_{k=1}^{k_m} c^k\right)\cdot a_1x_1 = r^*+c^*\cdot a_1x_1
\]

然后,敌手就可以进一步基于 \(s_1^*\) 构造 
\[
  s^* = s_1^* + c^*\cdot a_2x_2 = r^* + c^*\cdot (a_1x_1+a_2\cdot x_2)
\]

因此,\((R^*,s^*)\) 即为 \(m^*\) 的合法签名,其中签名哈希为 \(c^* = H_{sig}(\tilde{X},R^*,m^*)\)。
这里伪造的消息只对 \(X_1\) 和 \(X_2\) 聚合所得公钥 \(\tilde{X}\) 合法。显然,只要是把诚实签名方的公钥 \(X_1\) 和敌手的公钥集合聚合,攻击只需稍作调整伪造出合法的消息。

Wagner 算法的 \textbf{攻击复杂度} 为 \(O(k_m 2^{\log_2(p)/(1+\lfloor \log_2(k_m) \rfloor)})\)。虽然是次指数级别(非多项式级别),攻击对于常用参数和足够大的 \(k_m\) 是可操作的。例如,对于椭圆曲线常用的素数 \(p\approx 2^{256}\),\(k_m=128\) 能够将攻击复杂度降低到约 \(2^{39}\) 次操作,普通硬件都能实施此攻击。如果敌手能够开启更多会话,Benhamouda 等人~\cite{cryptoeprint:2020:945} 改进的多项式时间级别的攻击使用 \(k_m>\log_2 p\) 就能实现 \(O(k_m\log_2 p)\) 攻击复杂度,实际操作性贼强。

\section{MuSig2 方案}

相关论文参见~\cite{cryptoeprint:2020:1261,jonasnick2020MuSig2}。

\subsection{背景}
Drijvers~\cite{2019On} 或 Benhamouda~\cite{cryptoeprint:2020:945} 等人的攻击方式均是通过控制聚合的 nonce \(R_1^k R_2^k\)(等式~\ref{fake-R2} 的左手边)来控制签名的哈希。由于所有签名方都在第一轮交互结束时知道聚合的 nonce,不像 ~\cite{cryptoeprint:2018:068} 添加额外承诺轮的情况下防止敌手控制左手边的聚合 nonce 有点难。

左边不好弄的话,不妨换个思路,我们允许敌手控制等式左手边,但是防止他们控制等式的右手边。

MuSig2 方案的巧妙点在于让每个签名方 \(i\) 发送一个 nonces 列表 \(R_{i,1},\dots,R_{i,\nu}\quad (\nu\geq 2)\),以它们的线性组合 \(\hat{R}_i=\prod_{j=1}^{\nu}R_{i,j}^{b_j}\) 作为自己最终的 nonce,而不是之前的单个 nonce \(R_i\),其中 \( b_j=H_{non}(j,\tilde{X},(\prod_{i=1}^n R_{i,1},\dots,\prod_{i=1}^n R_{i,\nu}),m) \),\(H_{non}: \{0,1\}^*\rightarrow \bZ_p\) 是一个可看做随机预言机的哈希函数。

这样一来,每次敌手尝试不同的 \(R_2^k\),系数 \(b_1^k,\dots,b_{\nu}^k\) 都会随之变化,进而改变诚实签名方的 \(\hat{R}_1=\prod_{j=1}^{\nu}R_{1,j}^{b_j}\),最终改变等式~\ref{fake-R2} 右手边的 \(R^*=\prod_{k=1}^{k_m} \hat{R}_1^k\)。这也就确保了等式右手边不再是常量,破坏掉泛化生日问题的必要前提条件,Wagner 算法也就不再适用。

关于 \(\nu=1\) 的情形不可行的具体原因分析如下(\textcolor{red}{尚未搞懂})。

既然这样,是否可以回退到单个 nonce 的情况呢(\(\nu=1\))--只依赖系数 \(b_1\)?然而,敌手还是可以通过计算以下等式d抵消这个变换的效果

\[
  \sum_{k=1}^{k_m} \frac{H_{sig}(\tilde{X},(R_1^k)^{b_1^k},m^k)}{b_1^k} = H_{sig}(\tilde{X},R^*,m^*)  
\]

这样就解释了 \(\nu=2\) 的必要性,我们后续会证明固定 \(b_1=1\)(随机化其余系数 \(b_2,\dots,b_{\nu}\))是一个优化技巧,且不会损害安全性。

\subsection{具体算法}
\subsubsection{签名}
\paragraph{设定参数} 给定群 \((\bG,p,g)\),以及三个 \(\{0,1\}^*\) 到 \(\bZ_p\) 的哈希函数。
\paragraph{生成密钥} 生成随机私钥 \(x\leftarrow_{\$} \bZ_p\),计算相应公钥 \(X=g^x\)。
\paragraph{生成 nonce} 令 \((x_1,X_1)\) 表示特定签名方的公私钥对。对于 \(j\in\{1,\dots,\nu\}\),签名方生成随机数 \(r_{1,j}\leftarrow \bZ_p\),计算 \(R_{1,j}=g^{r_{1,j}}\),向其他所有签名方广播 \((R_{1,1},\dots,R_{1,\nu})\)。
\paragraph{生成签名碎片} 给定消息 \(m\),其他签名方的公钥为 \(X_2,\dots,X_n\),令 \(L=\{X_1,\dots,X_n\}\) 表示签名过程涉及的所有公钥。对于 \(i\in\{1,\dots,n\}\),签名方计算 \(a_i=H_{agg}(L,X_i)\),然后计算聚合公钥 \(\tilde{X}=\prod_{i=1}^n X_i^{a_i}\)。一旦\textbf{收齐}其他签名方的 \((R_{2,1},\dots,R_{2,\nu}),\dots,(R_{n,1},\dots,R_{n,\nu})\),计算 
\begin{align*}
  R_j &= \prod_{i=1}^n R_{i,j}\quad (j\in\{1,\dots,\nu\}) \\
  (b_1,\dots,b_{\nu}) &=\left(1,,H_{non}(2,\tilde{X},(R_1,\dots,R_{\nu}),\dots,H_{non}(\nu,\tilde{X},(R_1,\dots,R_{\nu})\right) \\
  R &= \prod_{j=1}^{\nu} R_j^{b_j} \Rightarrow c = H_{sig}(\tilde{X},R,m) \Rightarrow s_1 = ca_1x_1+\sum_{j=1}^{\nu}r_{1,j}b_j\bmod p
\end{align*}

把 \(s_1\) 发送给其他所有签名方。

\paragraph{聚合签名碎片} 收齐其他方的 \(s_2,\dots,s_n\) 之后,计算 \(s=\sum_{i=1}^s s_i\bmod p\),输出最终签名为 \(\sigma=(R,s)\)。

\subsubsection{验签}
给定公钥集合 \(L=\{X_1,\dots,X_n\}\),消息 \(m\) 和签名 \(\sigma=(R,s)\),验证方计算
\[
  a_i=H_{agg}(L,X_i) \quad (i\in\{1,\dots,n\}) \Rightarrow \tilde{X}=\prod_{i=1}^n X_i^{a_i} \Rightarrow c=H_{sig}(\tilde{X},R,m)
\]
 
如果 \(g^s=R\prod_{i=1}^n X_i^{a_i c}=R\tilde{X}\),则签名合法。

\subsection{与 MuSig 相比}
\cite{cryptoeprint:2018:068} 的签名需要三轮,而 MuSig2 只需要两轮。

% 更新 bibtex 需要执行脚本 renew_bibliography.sh
\bibliographystyle{alpha}
\bibliography{ref}

\end{document}