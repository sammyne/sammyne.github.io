%! TEX program=xelatex
\documentclass[a4paper,10pt]{article}

\usepackage[ruled,linesnumbered,vlined]{algorithm2e}
\usepackage{amssymb}
\usepackage{amsmath}
\usepackage{amsthm}
\usepackage{cite}
\usepackage{ctex}
\usepackage{datetime2}
\usepackage{enumitem}
\usepackage[top=1cm,bottom=2cm]{geometry}
\usepackage[colorlinks,
  linkcolor=blue,
  anchorcolor=blue,
  citecolor=blue
]{hyperref}
\usepackage{listings}
\usepackage{url}

\lstset{
  basicstyle=\small,
  keywordstyle=\ttfamily,
  identifierstyle=\ttfamily,
  stringstyle=\ttfamily,
  showstringspaces=false
}

\newcommand{\bG}{\mathbb{G}}
\newcommand{\bZ}{\mathbb{Z}}

\newtheorem{definition}{\bf 定义}[section]

\title{代理重加密(Proxy Re-Encryption)}
\date{\DTMdate{2020-11-18}}
\author{Sammy Li}

\begin{document}
\maketitle

\tableofcontents

\section{符号定义}
\begin{table}[!htbp]
  \centering
  \caption{符号定义}
  \begin{tabular}{|r|l|}
    \hline
    符号 & 说明 \\
    \hline
    \(\bG=\langle G\rangle\) & 椭圆曲线群,生成元记为 \(G\) \\
    \(e: \bG_1\times \bG_2\rightarrow\bG_T\) & 椭圆曲线群 \(\bG_1=\langle G_1\rangle\) 和 \(\bG_2=\langle G_2\rangle\) 到 \(\bG_T=\langle G_T\rangle\) 的双线性映射 \\
    \(s\in_R S\) & 表示从集合 \(S\) 随机选取一个元素 \(s\) \\ 
    \(p\) & 群 \(\bG_1\) 和 \(\bG_2\) 的阶,即元素个数 \\
    \hline
  \end{tabular} 
\end{table}

往后部分使用的具体椭圆曲线群为 bn256\cite{2005Pairing,2010New}。

\section{流程}
流程涉及角色如下
\begin{itemize}
  \item 发送方 A:负责本地加密消息,然后上传到代理服务器,为接收方分配重加密的解密密钥
  \item 代理服务器 S:存储 A 的密文,基于 A 的重加密密钥执行重加密
  \item 接收方 B:基于重加密密钥解密密文
\end{itemize}

\subsection{A 和 B 本地生成密钥}
\begin{itemize}
  \item A 本地基于 \(\bG_1\) 生成公私钥对 \((x_A,X_A=x_A\cdot G_1)\)
  \item B 本地基于 \(\bG_2\) 生成公私钥对 \((x_B,X_B=x_B\cdot G_2)\)
\end{itemize}

\subsection{A 加密消息并上传到 S}

\begin{enumerate}
  \item 将 \(m\) 映射为 \(\bG_T\) 的元素 \(M\)
    \begin{itemize}
      \item 通常情况下,\(m\) 用作加密数据的对称密钥,可先随机生成 \(M\) 然后转化为 \(m\),再用 \(m\) 对数据进行对称加密
    \end{itemize}
  \item 随机生成 \(r\in_R \bZ_p\)
  \item 计算密文分片
    \begin{itemize}
      \item \(C_1=r\cdot G_T+M \in\bG_T\)
      \item \(C_2=r\cdot X_A \in\bG_1\)
    \end{itemize}
  \item 将 \((C_1,C_2)\) 上传到 S
\end{enumerate}

\subsection{A 基于 B 的公钥生成重加密密钥并上传到 S}

\begin{enumerate}
  \item A 计算重加密密钥 \(X'=x_A'^{-1}\cdot X_B\in\bG_2\)
  \item 将 \(X'\) 上传到 S
\end{enumerate}

\subsection{S 加工生成重加密消息}

S 计算并保存 \(C_2'=e(X',C_2)\)

\subsection{B 执行重解密还原消息}

\begin{enumerate}
  \item B 从 S 下载 \((C_1,C_2')\)
  \item 计算 \(R=x_B'^{-1}\cdot C_2'\)
  \item 计算 \(M'=C_1-R\)
  \item 将 \(M'\) 转化为 \(m'\),即得消息原文 \(m=m'\)
\end{enumerate}

\section{证明}
\subsection{预备知识}
\subsubsection{双线性映射}
设\(\bG_1\)、\(\bG_2\) 和 \(\bG_T\) 都是阶为 \(p\) 的循环群,\(p\) 是素数。如果映射 \(e: \bG_1\times\bG_2\rightarrow\bG_T\) 满足以下性质:

\begin{enumerate}
  \item \textbf{双线性性}: 对于任意 \(a,b\in \bZ_p\) 和 \(X\in\bG_1, Y\in\bG_2\),有 \(e(a\cdot X, b\cdot Y) = a\cdot b\cdot e(X, Y)\)
  \item \textbf{非退化性}: 存在 \(X\in\bG_1, Y\in\bG_2\),使得 \(e(X,Y)\neq G_T\)。这里 \(G_T\) 代表 \(\bG_T\) 群的单位元
  \item \textbf{可计算性}: 存在有效的算法对任意的 \(X\in\bG_1, Y\in\bG_2\),计算 \(e(X,Y)\) 的值
\end{enumerate}

那么称 \(e\) 是一个双线性映射。

双线性映射可以通过有线域上的超椭圆曲线上的 Tate 对或 Weil 对来构造。

\textbf{注}:\(e(X,Y)=e(Y,X)\)。

\subsection{解密过程正确性推导}
\begin{align*}
  M' &= C_1-R \\
     &= r\cdot G_T + M - x_B'^{-1} \cdot C_2' \\
     &= r\cdot G_T + M - x_B'^{-1} \cdot e(X',C_2) \\
     &= r\cdot G_T + M - x_B'^{-1} \cdot e(x_A'^{-1}\cdot X_B,C_2) \\
     &= r\cdot G_T + M - x_B'^{-1} \cdot e(x_A'^{-1}\cdot X_B,r\cdot X_A) \\
     &= r\cdot G_T + M - x_B'^{-1} \cdot e(x_A'^{-1}\cdot x_B \cdot G_2,r\cdot x_A\cdot G_1) \\
     &= r\cdot G_T + M - e(x_B'^{-1}\cdot x_B \cdot G_2,x_A'^{-1} \cdot r\cdot x_A\cdot G_1) \\
     &= r\cdot G_T + M - e(G_2,r\cdot G_1) \\
     &= r\cdot G_T + M - r\cdot e(G_2,G_1) \\
     &= r\cdot G_T + M - r\cdot G_T \\
     &= M
\end{align*}

\phantomsection % 使得链接能正确跳转
\begin{thebibliography}{99}
  \addcontentsline{toc}{section}{参考文献} %向目录中添加条目,以章节的名义

  \bibitem{2005Pairing} Barreto, Paulo S. L. M. , and M. Naehrig . "Pairing-Friendly Elliptic Curves of Prime Order." (2005).
  \bibitem{2010New} Naehrig, Michael , R. Niederhagen , and P. Schwabe . "New Software Speed Records for Cryptographic Pairings." International Conference on Progress in Cryptology-latincrypt Springer-Verlag, 2010.
\end{thebibliography}

\end{document}